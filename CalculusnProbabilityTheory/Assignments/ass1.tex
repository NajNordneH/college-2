% Created 2021-04-10 Sat 18:08
% Intended LaTeX compiler: pdflatex
\documentclass[11pt]{article}
\usepackage[utf8]{inputenc}
\usepackage[T1]{fontenc}
\usepackage{graphicx}
\usepackage{grffile}
\usepackage{longtable}
\usepackage{wrapfig}
\usepackage{rotating}
\usepackage[normalem]{ulem}
\usepackage{amsmath}
\usepackage{textcomp}
\usepackage{amssymb}
\usepackage{capt-of}
\usepackage{hyperref}
\newcommand{\abs}[1]{\ensuremath{|#1| }}
\author{Jan Hendron(s1049777)}
\date{\today}
\title{Assignment 1}
\hypersetup{
 pdfauthor={Jan Hendron(s1049777)},
 pdftitle={Assignment 1},
 pdfkeywords={},
 pdfsubject={},
 pdfcreator={Emacs 27.1 (Org mode 9.5)},
 pdflang={English}}
\begin{document}

\maketitle

\section*{Exercise 6}
\label{sec:org279db86}
We must make some case distinctions in this question

\subsubsection*{a = 0}
When $a = 0$, $y = bx$, so the range of $y = (0,b)$ if $b > 0$, otherwise, it is $(b,0)$
\subsubsection*{b = 0}
When $b = 0$, $y = a - ax$, so the range of $y = (0,a)$ if $a > 0$, otherwise, it is $(a,0)$
\subsubsection*{a and b \neq 0 }
\begin{itemize}
\item
When $a > 0$ and $b = 0$, $y = a - ax$, and since $x$ is in the range of $(0,1)$, then $y$ must be in the range of $(0,a)$, as

When $x = 0$
\begin{align*}
  y &= a-a(0) \\
  y &= a
\end{align*}

When $x = 1$
\begin{align*}
  y &= a-a(1)\\
  y &= a-a \\
  y &= 0
\end{align*}

But since we have already said that $a > 0$, then the range of $y$ must be $(0,a)$
 \item
        When $a > 0$ and $b > 0$, $y = a + bx - ax$, and since $x$ is in the range of $(0,1)$, then if we substitute in values for $x$, we can see that

        $x = 0$

\begin{align*}
  y &= a+b(0)-a(0) \\
  y &= a
\end{align*}

        $x = 1$

\begin{align*}
  y &= a+b(1)-a(1) \\
  y &= a-a+b \\
  y &= b
\end{align*}
        So, if $a > b$, then the range of $y = (b,a)$, and this will also hold for when $b < 0$, and if $b > a$, then the range of $y = (b,a)$, and this will also hold for when $a < 0$

        If $a = b$, then the range will be $y = (a,a) = (b,b)$, which means that the function will just be a straight line on a graph
\end{itemize}\section*{Exercise 7}
\label{sec:org19c2a8d}
To prove that a function is odd, we must show that \(-f(x)=f(-x)\; \forall x\in D\), and to prove that a function is even, we must show that \(f(x) = f(-x)\; \forall x \in D\)

\begin{enumerate}

  \item[a)] \begin{align*}
\textrm{\(f(-x)\) should be equal to this if it is even}\\
f(x) &= 3x-x^{3} \\
& \\
\textrm{\(f(-x)\) should be equal to this if it is odd} \\
-f(x) &= -(3x-x^{3})\\
&= -3x+x^{3} \\
&= x^{3}-3x \\
&  \\
\textrm{Here we test if either it true} \\
f(-x) &= 3(-x)-(-x)^{3}   \\
&= -3x-(-x^3) \\
&= -3x+x^3 \\
&= x^3-3x
\end{align*}
Therefore, \(f(x) = 3x-x^{3}\) is an \textbf{odd} function

  \item[b)] \begin{align*}
\textrm{\(f(-x)\) should be equal to this if it is even} \\
 f(x) &= \sqrt[3]{(1-x)^{2}} + \sqrt[3]{(1+x)^2} \\
 \textrm{\(f(-x)\) should be equal to this if it is odd} \\
 -f(x) &= -\left(\sqrt[3]{(1-x)^2} + \sqrt[3]{(1+x)^2}\right)\\
 &= - \sqrt[3]{(1-x)^2} - \sqrt[3]{(1+x)^2}\\
 \textrm{Here we test if either are true}\\
 f(-x) &= \sqrt[3]{(1-(-x))^{2}} + \sqrt[3]{(1+(-x))^2} \\
 &= \sqrt[3]{(1+x)^{2}} + \sqrt[3]{(1-x)^2} \\
 &\textrm{Rearrange} \\
 &= \sqrt[3]{(1-x)^{2}} + \sqrt[3]{(1+x)^2} \\
\end{align*}
Therefore, \(f(x) = \sqrt[3]{(1-x)^{2}} + \sqrt[3]{(1+x)^2}\) is an \textbf{even} function
\end{enumerate}
\section*{Exercise 8}
\label{sec:org33e8865}
\begin{enumerate}
  \item[a)]
        The domain($D(f)$) of this function must be $\left[-\sqrt{7},\sqrt{7}\right]$, as if $x = 0$,  then the function is just $\sqrt{7}+1$, but $x$ can never be greater than $\sqrt{7}$, or less than $-\sqrt{7}$, as then we would have the square root of a negative number, which is not possible in the real plane, but $x$ can be equal to $\pm \sqrt{7}$, as $\sqrt{0} = 0$.

        The range($R(f)$) of the function must be $\left[1,1+\sqrt{7}\right]$, as when $x = 0$, $y = \sqrt{7}+1$, but when $x = \pm \sqrt{7}$,  then $y = 1$
  \item[b)]
        The domain($D(f)$) of this function is $\mathbb{R} - {0}$, as you can never divide by 0, but apart from that any real number can be used

        The range($R(f)$) of this function is $\left(0,\infty\right)$, as $\lim\limits_{x \rightarrow 0}\frac{1}{\abs{x}} = \infty$, because as $x$ gets smaller $y$ will get bigger, but since $\forall x\in D(f)\; \; x \neq 0$, it will never reach $\infty$, and  $\lim\limits_{x \rightarrow \infty}\frac{1}{\abs{x}} = 0$, and since the absolute value will ensure there is always a positive number as the denominator, therefore the function will never tend towards $-\infty$, so $R(f) = \left(0,\infty\right)$
\end{enumerate}
\section*{Exercise 9}
\label{sec:org9e11fd4}
\begin{enumerate}
\item In order to compute the inverse of a function \(f^{-1}(x)\), you have to express the function in terms of \(x\), and then change the variables, so from the function \(y = \frac{ax+b}{cx+d}\)

\begin{gather*}
y = \frac{ax+b}{cx+d} \\
y(cx+d) = ax+b \\
cxy + dy = ax+b \\
cxy - ax =b-dy \\
x(cy-a) = b-dy \\
x = \frac{b-dy}{cy-a}\\
y = \frac{b-dx}{cx-a}
\end{gather*}

\item
\end{enumerate}
\section*{Exercise 10}
\label{sec:org48b59b0}
\begin{enumerate}
\item[a)]
\begin{gather*}
\lim_{x\to2} \frac{x-2}{x^{2}+x-6} \\
\end{gather*}
This will be done using the \textbf{simplify} method

Factorize the denominator
\begin{gather*}
\frac{x-2}{(x-2)(x+3)}
\end{gather*}
Cancel the \((x-2)\) on both sides of the fraction
\begin{gather*}
\frac{1}{x+3} \\
\frac{1}{2+3} \\
\lim_{x\to2} \frac{x-2}{x^{2}+x-6} = \frac{1}{5} \\
\end{gather*}
  \item[b)]
        \textbf{Squeeze} theorem:

        Since we have $-1 \le cos \left(\frac{1}{x}\right) \le 1 $, then we must have that $\abs{x} \le - \abs{x}cos \left(\frac{1}{x}\right) \le \abs{x}$.

        Now, we know that the $\lim_{x \rightarrow 0} - \abs{x} = 0 = \lim_{x \rightarrow 0} \abs{x}$,  therefore $\lim_{x \rightarrow 0} \abs{x} cos \left(\frac{1}{x}\right) = 0$

  \item[c)]

        \[
        \lim_{x\rightarrow 1}\frac{x^{2}+4x+3}{x^{2}+x-2}
        \]

        I will use the \textbf{simplify} method:

\begin{gather*}
  \frac{x^{2}+4x+3}{x^{2}+x-2} \\
  \frac{(x-3)(x-1)}{(x+2)(x-1)} \qquad \textrm{Factorize} \\
  \frac{x-3}{x+2} \qquad \textrm{Cancel the \((x-1)\)} \\
  \frac{1-3}{1+3} \qquad \textrm{Substitute \(x = 1\)} \\
  \frac{-2}{4} \\
  - \frac{1}{2}
\end{gather*}


        \[
        \lim_{x\rightarrow 1}\frac{x^{2}+4x+3}{x^{2}+x-2} = - \frac{1}{2}
        \]
  \item[d)]
        \[
  \lim_{x \rightarrow 0} \frac{\sqrt{x^{2}+x+1}-\sqrt{x^{2}+1}}{x}
        \]

        I will use the \textbf{rationalize} method here:


\begin{gather*}
  \\
  \frac{\sqrt{x^{2}+x+1}-\sqrt{x^{2}+1}}{x} \times \frac{{\sqrt{x^{2}+x+1}+\sqrt{x^{2}+1}}}{\sqrt{x^{2}+x+1}+\sqrt{x^{2}+1}} \\
  \\
  \frac{x^{2}+x+1-x^{2}-1}{x\left(\sqrt{x^{2}+x+1}+\sqrt{x^{2}+1}\right)} \\
  \\
  \frac{x}{x\left(\sqrt{x^{2}+x+1}+\sqrt{x^{2}+1}\right)} \\
  \\
  \frac{1}{\left(\sqrt{x^{2}+x+1}+\sqrt{x^{2}+1}\right)}
\end{gather*}
        Substitute \(x = 0\)
\begin{gather*}
  \frac{1}{\sqrt{0^{2}+0+1}+\sqrt{0^{2}+1}} \\
  \\
  \frac{1}{\sqrt{1}+\sqrt{1}} \\
  \\
  \frac{1}{1+1}\\
  \\
  \frac{1}{2} \\
\end{gather*}
        So,
        \[
  \lim_{x \rightarrow 0} \frac{\sqrt{x^{2}+x+1}-\sqrt{x^{2}+1}}{x} = \frac{1}{2}
        \]
\end{enumerate}
\section*{Exercise 11}
First, we should find what $f^{-1}(x)$ is
\begin{align*}
  y &= \frac{x}{2x+3} \\
  y(2x+3) &= x \\
  2xy +3y &= x \\
  3y &= x-2xy \\
  3y &= x(1-2y) \\
  \frac{3y}{1-2y} = x \\
  f^{-1}(x) = \frac{3x}{1-2x}
\end{align*}
So, to find $a$ and $b$,  we must find a line that connects the points $(-2,-\frac{6}{5})$, as this is the last point on the left that $g(x) = f^{-1}(x)$, and $(3,-\frac{9}{5})$.

First, we must find the slope of the line
\begin{align*}
\frac{dy}{dx} &= \frac{-\frac{6}{5}+\frac{9}{5}}{-2-3} \\
              &= \frac{\frac{3}{5}}{-5} \\
  &= - \frac{3}{25}
\end{align*}

Because the equation of the line was give as $ax+b$, we know now that $a = -\frac{3}{25}$, so we can solve for $b$
\begin{gather*}
  -\frac{3}{25} \cdot (-2) + b = -\frac{6}{5} \\
  \frac{6}{25}+b = -\frac{6}{5} \\
  b= -\frac{6}{5} - \frac{6}{25} \\
  b = -\frac{36}{25} \\
\end{gather*}

So, we can now see that
\[
  \frac{b}{a} = \frac{-36}{-3} = 12
\]

\end{document}
